%%%%%%%%%%%%%%%%%%%%%%%%%%%%%%%%%%%%%%%%%%%%%%%%%%%%%%%%%%%%%%%%%%%%%%
%%  disstemplate.tex, to be compiled with latex.		     %
%%  08 April 2002	Version 4				     %
%%%%%%%%%%%%%%%%%%%%%%%%%%%%%%%%%%%%%%%%%%%%%%%%%%%%%%%%%%%%%%%%%%%%%%
%%								     %
%%  Writing a Doctoral Dissertation with LaTeX at		     %
%%	the University of Texas at Austin			     %
%%								     %
%%  (Modify this ``template'' for your own dissertation.)	     %
%%								     %
%%%%%%%%%%%%%%%%%%%%%%%%%%%%%%%%%%%%%%%%%%%%%%%%%%%%%%%%%%%%%%%%%%%%%%

\documentclass[12pt]{report}	% The documentclass must be ``report''.

\usepackage{utdiss2}  		% Dissertation package style file.


%%%%%%%%%%%%%%%%%%%%%%%%%%%%%%%%%%%%%%%%%%%%%%%%%%%%%%%%%%%%%%%%%%%%%%
% Optional packages used for this sample dissertation. If you don't  %
% need a capability in your dissertation, feel free to comment out   %
% the package usage command.					     %
%%%%%%%%%%%%%%%%%%%%%%%%%%%%%%%%%%%%%%%%%%%%%%%%%%%%%%%%%%%%%%%%%%%%%%

\usepackage{amsmath,amsthm,amsfonts,amscd} 
				% Some packages to write mathematics.
\usepackage[dvipsnames]{xcolor}
\usepackage{eucal} 	 	% Euler fonts
\usepackage{verbatim}      	% Allows quoting source with commands.
\usepackage{makeidx}       	% Package to make an index.
\usepackage{booktabs}
% \usepackage{psfig}         	% Allows inclusion of eps files.
\usepackage{epsfig}         	% Allows inclusion of eps files.
% \usepackage{citesort}         	% 
\usepackage{url}		% Allows good typesetting of web URLs.
\usepackage{draftcopy}		% Uncomment this line to have the
				% word, "DRAFT," as a background
				% "watermark" on all of the pages of
				% of your draft versions. When ready
				% to generate your final copy, re-comment
				% it out with a percent sign to remove
				% the word draft before you re-run
				% Makediss for the last time.

\author{Christopher Edwin Crabtree}  	% Required

\address{110 Charles St\\ Rockville, Maryland 20850}  % Required

\title{Optimizing Visual Grounding of Latent Representations of Speech 
from Distant Language Groups}

                                                    % Required

%%%%%%%%%%%%%%%%%%%%%%%%%%%%%%%%%%%%%%%%%%%%%%%%%%%%%%%%%%%%%%%%%%%%%%
% NOTICE: The total number of supervisors and other members %%%%%%%%%%
%%%%%%%%%%%%%%% MUST be seven (7) or less! If you put in more, %%%%%%%
%%%%%%%%%%%%%%% they are put on the page after the Committee %%%%%%%%%
%%%%%%%%%%%%%%% Certification of Approved Version page. %%%%%%%%%%%%%%
%%%%%%%%%%%%%%%%%%%%%%%%%%%%%%%%%%%%%%%%%%%%%%%%%%%%%%%%%%%%%%%%%%%%%%

%%%%%%%%%%%%%%%%%%%%%%%%%%%%%%%%%%%%%%%%%%%%%%%%%%%%%%%%%%%%%%%%%%%%%%
%
% Enter names of the supervisor and co-supervisor(s), if any,
% of your dissertation committee. Put one name per line with
% the name in square brackets. The name on the last line, however,
% must be in curly braces.
%
% If you have only one supervisor, the entry below will read:
%
%	\supervisor
%		{Supervisor's Name}
%
% NOTE: Maximum three supervisors. Minimum one supervisor.
% NOTE: The Office of Graduate Studies will accept only two supervisors!
% 
%
\supervisor
	% [Isaac Newton]
	{David Harwath}

%%%%%%%%%%%%%%%%%%%%%%%%%%%%%%%%%%%%%%%%%%%%%%%%%%%%%%%%%%%%%%%%%%%%%%
%
% Enter names of the other (non-supervisor) members(s) of your
% dissertation committee. Put one name per line with the name
% in square brackets. The name on the last line, however, must
% be in curly braces.
%
% NOTE: Maximum six other members. Minimum zero other members.
% NOTE: The Office of Graduate Studies may restrict you to a total
%	of six committee members.
%
%
\committeemembers
	% [Erwin Schr\"odinger]
	% [Albert Einstein]
	% [Charles Townes]
	{Greg Durrett}

%%%%%%%%%%%%%%%%%%%%%%%%%%%%%%%%%%%%%%%%%%%%%%%%%%%%%%%%%%%%%%%%%%%%%%

\previousdegrees{B.S.}
     % The abbreviated form of your previous degree(s).
     % E.g., \previousdegrees{B.S., MBA}.
     %
     % The default value is `B.S., M.S.'

%\graduationmonth{...}      
     % Graduation month, either May, August, or December, in the form
     % as `\graduationmonth{May}'. Do not abbreviate.
     %
     % The default value (either May, August, or December) is guessed
     % according to the time of running LaTeX.

%\graduationyear{...}   
     % Graduation year, in the form as `\graduationyear{2001}'.
     % Use a 4 digit (not a 2 digit) number.
     %
     % The default value is guessed according to the time of 
     % running LaTeX.

%\typist{...}       
     % The name(s) of typist(s), put `the author' if you do it yourself.
     % E.g., `\typist{Maryann Hersey and the author}'.
     %
     % The default value is `the author'.


%%%%%%%%%%%%%%%%%%%%%%%%%%%%%%%%%%%%%%%%%%%%%%%%%%%%%%%%%%%%%%%%%%%%%%
% Commands for master's theses and reports.			     %
%%%%%%%%%%%%%%%%%%%%%%%%%%%%%%%%%%%%%%%%%%%%%%%%%%%%%%%%%%%%%%%%%%%%%%
%
% If the degree you're seeking is NOT Doctor of Philosophy, uncomment
% (remove the % in front of) the following two command lines (the ones
% that have the \ as their second character).
%
%\degree{MASTER OF ARTS}
%\degreeabbr{M.A.}

% Uncomment the line below that corresponds to the type of master's
% document you are writing.
%
%\masterreport
%\masterthesis


%%%%%%%%%%%%%%%%%%%%%%%%%%%%%%%%%%%%%%%%%%%%%%%%%%%%%%%%%%%%%%%%%%%%%%
% Some optional commands to change the document's defaults.	     %
%%%%%%%%%%%%%%%%%%%%%%%%%%%%%%%%%%%%%%%%%%%%%%%%%%%%%%%%%%%%%%%%%%%%%%
%
%\singlespacing
%\oneandonehalfspacing

%\singlespacequote
\oneandonehalfspacequote

\topmargin 0.125in	% Adjust this value if the PostScript file output
			% of your dissertation has incorrect top and 
			% bottom margins. Print a copy of at least one
			% full page of your dissertation (not the first
			% page of a chapter) and measure the top and
			% bottom margins with a ruler. You must have
			% a top margin of 1.5" and a bottom margin of
			% at least 1.25". The page numbers must be at
			% least 1.00" from the bottom of the page.
			% If the margins are not correct, adjust this
			% value accordingly and re-compile and print again.
			%
			% The default value is 0.125"

		% If you want to adjust other margins, they are in the
		% utdiss2-nn.sty file near the top. If you are using
		% the shell script Makediss on a Unix/Linux system, make
		% your changes in the utdiss2-nn.sty file instead of
		% utdiss2.sty because Makediss will overwrite any changes
		% made to utdiss2.sty.

%%%%%%%%%%%%%%%%%%%%%%%%%%%%%%%%%%%%%%%%%%%%%%%%%%%%%%%%%%%%%%%%%%%%%%
% Some optional commands to be tested.				     %
%%%%%%%%%%%%%%%%%%%%%%%%%%%%%%%%%%%%%%%%%%%%%%%%%%%%%%%%%%%%%%%%%%%%%%

% If there are 10 or more sections, 10 or more subsections for a section,
% etc., you need to make an adjustment to the Table of Contents with the
% command \longtocentry.
%
%\longtocentry 



%%%%%%%%%%%%%%%%%%%%%%%%%%%%%%%%%%%%%%%%%%%%%%%%%%%%%%%%%%%%%%%%%%%%%%
%	Some math support.					     %
%%%%%%%%%%%%%%%%%%%%%%%%%%%%%%%%%%%%%%%%%%%%%%%%%%%%%%%%%%%%%%%%%%%%%%
%
%	Theorem environments (these need the amsthm package)
%
%% \theoremstyle{plain} %% This is the default

\newtheorem{thm}{Theorem}[section]
\newtheorem{cor}[thm]{Corollary}
\newtheorem{lem}[thm]{Lemma}
\newtheorem{prop}[thm]{Proposition}
\newtheorem{ax}{Axiom}

\theoremstyle{definition}
\newtheorem{defn}{Definition}[section]

\theoremstyle{remark}
\newtheorem{rem}{Remark}[section]
\newtheorem*{notation}{Notation}

%\numberwithin{equation}{section}


%%%%%%%%%%%%%%%%%%%%%%%%%%%%%%%%%%%%%%%%%%%%%%%%%%%%%%%%%%%%%%%%%%%%%%
%	Macros.							     %
%%%%%%%%%%%%%%%%%%%%%%%%%%%%%%%%%%%%%%%%%%%%%%%%%%%%%%%%%%%%%%%%%%%%%%
%
%	Here some macros that are needed in this document:

\newcommand{\V}[1][i]{\mathcal{V}_#1}
\newcommand{\D}{\mathcal{D}}
\renewcommand{\th}{^{th}}
\newcommand{\dftwrds}[1][INSERT LATER]{\textcolor{red}{#1}}
\newcommand{\citeme}[1][CITEME]{\textcolor{red}{(#1)}}

\newcommand{\latexe}{{\LaTeX\kern.125em2%
                      \lower.5ex\hbox{$\varepsilon$}}}

\newcommand{\amslatex}{\AmS-\LaTeX{}}

\chardef\bslash=`\\	% \bslash makes a backslash (in tt fonts)
			%	p. 424, TeXbook

\newcommand{\cn}[1]{\texttt{\bslash #1}}

\makeatletter		% Starts section where @ is considered a letter
			% and thus may be used in commands.
\def\square{\RIfM@\bgroup\else$\bgroup\aftergroup$\fi
  \vcenter{\hrule\hbox{\vrule\@height.6em\kern.6em\vrule}%
                                              \hrule}\egroup}
\makeatother		% Ends sections where @ is considered a letter.
			% Now @ cannot be used in commands.

\makeindex    % Make the index

%%%%%%%%%%%%%%%%%%%%%%%%%%%%%%%%%%%%%%%%%%%%%%%%%%%%%%%%%%%%%%%%%%%%%%
%		The document starts here.			     %
%%%%%%%%%%%%%%%%%%%%%%%%%%%%%%%%%%%%%%%%%%%%%%%%%%%%%%%%%%%%%%%%%%%%%%

\begin{document}

\copyrightpage          % Produces the copyright page.


%
% NOTE: In a doctoral dissertation, the Committee Certification page
%		(with signatures) is BEFORE the Title page.
%	In a masters thesis or report, the Signature page
%		(with signatures) is AFTER the Title page.
%
%	If you are writing a masters thesis or report, you MUST REVERSE
%	the order of the \commcertpage and \titlepage commands below.
%
\commcertpage           % Produces the Committee Certification
			%   of Approved Version page (doctoral)
			%   or Signature page (masters).
			%		20 Mar 2002	cwm

\titlepage              % Produces the title page.



%%%%%%%%%%%%%%%%%%%%%%%%%%%%%%%%%%%%%%%%%%%%%%%%%%%%%%%%%%%%%%%%%%%%%%
% Dedication and/or epigraph are optional, but must occur here.      %
%%%%%%%%%%%%%%%%%%%%%%%%%%%%%%%%%%%%%%%%%%%%%%%%%%%%%%%%%%%%%%%%%%%%%%
%
\begin{dedication}
\index{Dedication@\emph{Dedication}}%
Dedicated to my wife Shirley.
\end{dedication}


\begin{acknowledgments}		% Optional
\index{Acknowledgments@\emph{Acknowledgments}}%
I wish to thank the multitudes of people who helped me. Time would
fail me to tell of \ldots
\end{acknowledgments}


% The abstract is required. Note the use of ``utabstract'' instead of
% ``abstract''! This was necessary to fix a page numbering problem.
% The abstract heading is generated automatically.
% Do NOT use \begin{abstract} ... \end{abstract}.
%
\utabstract
\index{Abstract}%
\indent
This document has the form of a ``fake'' doctoral dissertation
in order to provide an example of such, but it is actually a
copy of Miguel Lerma's documentation for the Mathematics
Department Computer Seminar of 25 March 1998 updated in July 2001
and following by Craig McCluskey to meet the March 2001
requirements of the Graduate School.

This document and its source file show to write a Doctoral Dissertation using 
\LaTeX{} and the utdiss2 package. 



\tableofcontents   % Table of Contents will be automatically
                   % generated and placed here.

\listoftables      % List of Tables and List of Figures will be placed
\listoffigures     % here, if applicable.



%%%%%%%%%%%%%%%%%%%%%%%%%%%%%%%%%%%%%%%%%%%%%%%%%%%%%%%%%%%%%%%%%%%%%%
% Actual text starts here.					     %
%%%%%%%%%%%%%%%%%%%%%%%%%%%%%%%%%%%%%%%%%%%%%%%%%%%%%%%%%%%%%%%%%%%%%%
%
% Including external files for each chapter makes this document simpler,
% makes each chapter simpler, and allows for generating test documents
% with as few as zero chapters (by commenting out the include statements).
% This allows quicker processing by the Makediss command file in case you
% are not working on a specific, long and slow to compile chapter. You
% can even change the chapter order by merely interchanging the order
% of the include statements (something I found helpful in my own
% dissertation).
%
\chapter{Introduction}
\index{Introduction@\emph{Introduction}}%

Recent years have seen an increasing research interest into using multi-modal grounding techniques to bolster classic natural language processing (NLP) and automated speech recognition (ASR) tasks.
% Grounded MT
Among these have been efforts to use information from visual modalities such as images to improve the performance of neural machine translation (MT) and ASR models \cite{ huang2020unsupervised, srinivasan2020fine, harwath2018interlingua,ohishi2020trilingual}.
These research efforts are often motivated by the intuition that humans most often learn language through processes that involve resolving relations and co-occurences between a source modality (most typically spoken language), with that of input from sensory obtained from their environment.

Therefore it is a reasonable hypothesis to assert that the addition of co-occurring contextual information from modalities such as vision would be likely to aid in many natural language understanding tasks. 
However, Calgayan et al. \cite{caglayan2019probing} noted that research efforts in multimodal machine translation (MMT) have not yet clearly shown that that the addition of co-occurring visual information substantially improves MT performance.
Their work did show however, that when linguistic information is scarce or noisy, models using visual information are better able to recover from the missing information during translation.
Additionally, Huang et al. \cite{huang2020unsupervised} showed that in unsupervised MT settings they were able to incorporate the visual modality to consistently improve translation without aligned corpi for training.
These results would seem to indicate that latent visual representations \textit{are} able to, in some fashion, improve alignments of the internal representations of sentences in different languages, but the mechanism by which this occurs is not well understood.

%ASR 
Similarly in the ASR domain, Srinivasan et al. \cite{srinivasan2020fine} found that while visual information did not necessarily improve performance of speech recognition on clean datasets, when degrading the audio input using word masking their novel neural architecture was better able to  recover the original word when leveraging features from a visual modality.

These empirical observations seem to suggest that, contrary to intuition, additional information (in the form of co-occurring modalities carrying mutual semantic information) does not necessarily substantially effect neural models' ability to recognize and utilize the patterns present in the primary source modality.

% Translational knowledge transfer
Parallel lines of research, though, into multilingual machine translation have shown that the addition of new language pairs \textit{does} tend to improve transnational performance of all models. 
This phenomena has been termed transnational knowledge transfer \cite{dabre2020comprehensive} and  was first shown by Johnson et al.\cite{johnson2017google}.
They used a single neural model to perform 12-way multilingua translation showing improvements in several language pairs over conventional models trained on a single language pair.
These results align, very generally speaking, with the intuition that richer information sources should enable pattern recognition algorithms, such as neural networks, to more quickly lock on the tiny variations in the input that are semantically relevant. 

% Interlingua paper and Trilingual embeddings show mutual embedding space improves performance
    % MT and Google's zero-shot paper shows
This line of reasoning is further bolstered by results by Harwath et al. \cite{harwath2018interlingua} which showed that the addition of visual context (in the form of pre-trained image features) approximately doubled the recall scores of cross-lingual audio retrieval between Hindi and English image descriptions.
Ohishi et al. \cite{ohishi2020trilingual} extended this line of research by showing that retrieval scores between audio-visual and visual-audio modalities improved for all pairs when augmenting English and Hindi Places 205 \cite{zhou2014learning} image caption dataset with Japanese descriptions.

% These results are especially encouraging considering the lack of lexical and syntactic overlap between English, Hindi, and Japanese.
% Although it is notoriously difficult to compare translation results between different \citeme{runner up paper from ACL '20} language pairs, at least among English speakers there is a well established hierarchy that delineates which languages are most difficult for the average English speaker to learn \citeme{DOD list}.
% While there could be a multitude of socio-economic reasons for this difference in difficulty that one might expect a machine learning agent to be invariant to, there is also a clear correlation between this difficulty ranking and the syntactic, lexical, and chronological separation of these languages from English.



% ST and DLPs - Reason why it might work

As of yet it remains unclear, though, what extent visual grounding might aid in ST and MT, though, or what might be the best way to incorporate visual information into these tasks.

% Intoduce task
% Retrieval
% TODO: Look at prev papers for task motivation
In this work we investigate the ability of modern neural architectures to produce semanticly  aligned  embedding spaces. 
We do this by assessing our models' performance on spoken image and multi-lingual utterance retrieval as a proxy for mutual alignment.
Concretely, the task of spoken image retrieval  is to take a spoken utterance that describes an image and retrieve the corresponding image from a pool of candidate images (in our case the pool is the entire annotated dataset).
The dataset we use contains three descriptions of each image, each description in one of three distinct languages: English, Hindi, or Japanese.
Likewise in cross-lingual utterance retrieval, the task is to retrive an utterance in a target language that describes the same image as an utterance in a given source language utterance.
% This task does not use the mutual image at inference time, but does not prohibit the use of the shared image during training.

% Task specifics
It is worth emphasizing that in this task and in our dataset there is no assumption of lexical or syntactic alignment between language utterances.
That is, speakers may choose to describe the same image in completely different manners, 
using words and expressions that do not directly align with what was spoken in another language about the same image.
This lack of syntactic alignment means that lexical and syntactic co-occurence information will be largely inconsistent or even absent.
Nevertheless, since every utterance share the same grounded reference there should intuitively be consistent mutual semantic information shared amongst each training image and utterance triplet.

The task of retrieval not only  serves as a proxy for latent  representation alignment but also has more immediate practical applications.
Using speech to describe an image is a more natural interface than text when describing visual phenomena.
However, speech recognition systems requires a large amount of hand engineering of many sub-components.
These include, but are not limited to: acoustic, pronunciation, and language models. 
Predictive distributions between each sub-component are often combined using a large finite state transducer.
% The task of creating a new ASR system is often a laborious task that can be both time and data intensive.
Furthermore, once an ASR system is built, at least using current technology, the highly tuned sub-components cannot be repurposed into submodules of ASR systems for other languages. 
Thus an entirely new system must be built for each new language.

This work explores instead the extent to which latent feature representations from distant languages can be embedded into the same semantic vector space.
End-to-end (E2E) neural network encoders are used to produce these representations, thereby eliminating much of the sub-component specific implementation overhead needed to build classic ASR models for retrieval.
Embedding all languages in the same space also provides the potential benefits of translational transfer learning effects described above.

There are still many aspects of E2E ASR models, though, that are not well understood, especially in multi-lingual settings.
In particular, the exact translational mechanism in modern neural models is difficult pinpoint so it is unclear whether mutually aligned semantic embedding spaces are a necessary component of a multi-lingual translation model.
Also, as Tschannen et al. \cite{tschannen2019mutual} has noted, even the general task of optimizing mutual information (MI) has been shown to have deteriorating and even degrading returns in downstream tasks.

This lack of clear understanding motivates the experiments presented in this work.
In particular, this work attempts to address three questions:

%ST 
% Several previous works have attempted to perform speech translation (ST) by directly encoding source speech and producing target text, but as of yet it
% Some of these models have recently demonstrated surprising effective translations, even rivaling classic text-to-text MT models, but the most promising results so far have been isolated to relatively similar language pairs (\citeme{give example}).


% Problems
% Issues addressed
% \textit{Goal}
\begin{enumerate}
% 1. Loss func exploration
    \item What type of learning objective (i.e. loss function) results in the best alignment of image and multi-lingual utterances as measured by average retrieval scores?
% 2. Loss function complexity
    \item Aligning multiple modalities in the same representation space suffers a quadratic increase in loss terms with the pairing each new modality, which can complicate optimization.
To what extent can this be mitigated?

% 3. Parameter efficiency
\item Similarly, training an entirely new encoder for each model is potentially wasteful.
    %it is reasonable to assume that at least some of the task of modelling linguistic information might be shared in internal model parameters (e.g. recognizing phonemes shared between languages), which is one explanation for results demonstrated in previous work on translational knowledge transfer. 
    Can model visual grounding aid parameter when multiple languages share encoder paramters?
\end{enumerate}

% The rest of this thesis is organized as follows: chapter \ref{} presents background information on the techniques used for the experiments.
% Chapter \ref{} summarizes the base architecture used.
% Chapter \ref{} details the dataset used and precise design of the experiments and section \ref{} discusses the results.
% The thesis concludes in chapter \ref{}.







% % Outline
% % MT
% Robust and effective machine translation (MT) has been a goal of natural language processing (NLP) researchers since the 1950s \citeme.
% % ST
% As language is most naturally represented by humans in the form of speech, there is also a long line of research into speech translation (ST), beginning with the pioneering work of \citeme.
%
% Early ST models were characterized by a cascade of specialized submodels, each engineered to to solve a smaller task decomposed from the full ST task. 
% Typically, this decomposition took the form of: first transcribing source speech into text,
% followed by using standard MT systems to translate source language text into the target language.
% More recently, there has been growning intrest in directly translating source language speech into target text in an end-to-end fashion (E2E ST).
% This has the advantages of retaining the information present in the speech signal that may be lost in the process of transcription.
% These include cues in the pitch and volume of speech that might indicate speaker emotion and intention. 
% E2E ST also benefits from the absence of error propogation that can plague cascaded ST systems.
%
% % Current problems
% Despite these benefits, E2E ST systems have failed to completely overtake their cascaded counterparts.
% As of this report, E2E ST systems have been shown to improve on cascaded systems only in high-resource settings with similar language pairs.
% Several works have shown that, following similar trends in text MT, languages with markedly dissimilar syntax \dftwrds{think of additional} (and in this sense `distant') have proven difficult to model effectively.
% Compared to MT settings, though, there are relatively few datasets for ST using distant languages which hamstrings the data hungry E2E systems.
% \dftwrds{CHECK THIS} Even when the amount training data is held constant, poor perfomance amongst distant language pairs (DLPs) persists. 
% This highlights the difficulty of the task and gives evidence that more data is necessary for existing techniques to model DLPs.
%
% One 
%% MT inherent
%%% Distant Langauge Pairs
%%% Data and model efficiency
%% ST inherent
%%% Error propogation
%%% Implementation overhead
%%% End-to-end models promise to alleviate, but data efficiency is problem

% Grounding can hopefully alleviate
% Why
%% It contains co-occurence information that is shared amongst language modalities 

% Immediate issues

% Image grounding with distant multi-lingual speech is largely unexplored

%% This work we explore methods to improve alignment of intermediat representations of modalities
%% Introduce vocab: modalities, views
%%% Why?
% Immediate application
% 







\chapter{Background}
\index{Backgroud}
This section will review the primary concepts that will be evaluated empirically through the experiments in Chapters \ref{chapter:objective_exploration}, \ref{chapter:loss_complexity}, and \ref{chapter:param_eff}

\section{Task Description}
\label{section:task_defs}
All experiments in this work are designed to estimate predictive performance on retrieval tasks.
This section formalizes the retrieval task and describes it in detail for the sake of clarity.
We also define terms that well be used for the rest of the paper.
Readers familiar with retrieval tasks are encouraged only to note the terminology we define in this section.

% First we will formally define terms, though, in the fully generalized setting of our task.
We are given a \underline{${\D}$}ataset, $\D$, containing a set of $K$ distinct information sources. 
We will refer to each information ;ource as a \textit{\underline{$\mathcal{V}$}iew}.
Section \ref{section:multiview_framework} discusses this convention further.

% These views are distinguishable, of course, but not necessarily ordered.
We will refer to an arbitrary view as $\V[j]$ where $j\in[1,K]$ indexes each view.
Each view, $\V[j]$,  contains $N$ samples and each sample, $\nu_{ij} \in \V[j], i\in [1,N], j\in[1,K]$ has a corresponding entry in each of the other $K-1$ views which share some sort of semantic relatedness.
Furthermore, each $V_j$ has it's own representation scheme for it's constituent samples making a conventional tensor representation with fixed dimensions for all samples in $D$ inexact.
This gives the generalized dataset for our task the following definition:
\begin{equation*}
    \D = \{(\nu_{i1}, \nu_{i2}, \cdots, \nu_{iK}) : \nu_{i1} \in \V[1], \nu_{i2} \in \V[2], \cdots, \nu_{iK} \in \V[K], \quad i \in [1,N]\}
\end{equation*}
To reduce ambiguity, we will refer to an arbitrary K-tuple in $\D$ as a \textit{datapoint} and an arbitrary member of a K-tuple/datapoint, $\nu_{ij}$, as a \textit{viewpoint}. 
We will be sure to clarify in cases where this definition of viewpoint allows for ambiguity between the colloquial use of viewpoint by instead using `point of view' for the colloquial sense as needed.

To be more concrete for a moment, in the dataset used for this work we have $K=4$ views in our dataset which are: 1) the set of images, 2) the English descriptions, 3) the Hindi descriptions, and 4) the Japanese descriptions (although we do not assume that particular ordering).
For each image (viewpoint) in the image view, there is a corresponding spoken utterance (viewpoint) describing that image in all three languages.
We therefore assume some type of reliable mutual semantic information between each viewpoint in all datapoints.

Also note that, on occasion, we further categorize our views based on \textit{modality}.
To be clear, this dataset contains two modalities: vision and audio; one view is in the vision modality (images) and three views are in the audio modality (English, Japanese, and Hindi utterances).
We therefore commonly differentiate retrieval results based on modality pairing, retrieval results from vision-audio pairs being referred to as `image retrieval' and audio-audio retrieval results referred to as cross-lingual.
This partitioning is made to enable readers to more clearly assess the the models' ability to capture the mutual information shared by each view.


As an aside, \textit{mutual information} (MI) has a well-known and well-defined mathematical form, but we will be using the term in the idiomatic sense.
Recent work by Tschannen et al. \cite{tschannen2019mutual} has shown that  many popular lower-bound maximization schemes for estimating MI break down in practice and do not correlate with expected performance gains.
Furthermore, MI is only well-studied in the two variable (view) case. 
The generalization of MI to three or more variables is referred to as \textit{interaction information} and has certain properties (such as permitting negative values) that make it difficult to interpret.


Keeping these points in mind, we still wish to encode the datapoints from each view into a shared vector space. 
Our task, then, in words is: given an input (indexed by i), $\nu_{ij}$, from the $j\th$ view, $V_j$, and a set of viewpoints from a target view $V_t$, we would like like to retrieve $\nu_{it}$ from $V_t$.
In order to do this we use an \textit{encoder} functions $f_{\theta_i}$ and $f_{\theta_t}$ to produce a numeric representation of the viewpoints from the source and target views respectively.
In this work all encoders take the form of neural networks, and in particular convolutional neural networks (CNNs).

The retrieval is then performed by comparing the input, $\nu_{ij}$, with all other $\nu_{ik} \in V_k$ and returning the $\nu_{ik}$ with the highest \textit{similarity score}.
This similarity score is measured by a chosen \textit{similarity function}, $S$.
$S$ follows the typical mathematical definition of a distance measure (i.e. it takes two arguments, it's always non-negative, and it is symmetric in it's arguments).
For retrieval tasks this similarity function typically takes one of the following forms: the dot-product, the euclidean distance, or the cosine similarity.

With all terms defined, we can write our most general leaning objective as:
\begin{equation}
    \underset{\theta}{\max} \Big(\sum_{(\nu_i1,\cdots, \nu_ik)\in D, i\in[1,N]} \;\; \sum_{j\in[1,K]} \underset{t\in[1,K], t\neq j }{\max} S(f_{\theta_j}(\nu_{i,j}), f_{\theta_t}(\nu_{i,t}, \,t))\Big)
\end{equation}

This optimization objective has no known exact solution and must be approximated.
Most commonly, this type of optimization problem is approximated using a differentiable loss function and a gradient-descent based optimization algorithm.
There are a number of loss-function in common use for this type of maximization objective, many of which are motivated by maximizing the estimate of the aforementioned MI estimate.
Which of these loss functions produces optimal results for our particular setting is the subject of the experiments in Chapter \ref{chapter:objective_exploration}.

\section{Loss Functions}
\label{chapter:background|section:loss_functions}
There are numerous loss functions that are designed to encourage `similarity' among pairs or groups of datapoints.
We explore a subset of these in this work an describe the important aspects of each below.
\subsection{Triplet}
\label{section:triplet}
Triplet loss has perhaps the longest history of use in this task.
Harwath et al. \cite{harwath2015deep} first introduced the task of spoken image retrieval and used a maximum margin objective between each final unpooled image embedding and final sequence embedding of paired images and utterances.
This obective eventually became formulated a the the triplet loss.

Generally, triplet losses take an \textit{anchor} embedding/vector $A$, \textit{truthy} embedding $T$ (which is positively associated with the anchor), and a \textit{falsey} embedding $F$, also known as an imposter, that is not associated with the anchor. 
Then, given a chosen similarity function, the triplet loss can be calculated as:
\begin{equation}
\label{orig_triplet_loss}
    Triplet(A, T, F) = \max(0, S(A, F) - S(A, T) +M)
\end{equation}
Where $M$ is some margin to be chosen a hyperparameter (often just 1). 
This loss effectively pushes the $A$ and $T$ embeddings toward each other and the $A$ and $F$ embeddings away from each other.
The margin, then, influences how stronly model is encouraged to push or pull the associted embeddings.

This was formalized in \cite{harwath2017unsupervised} for spoken image retrieval by using two triplet losses: one defined with image representaions as anchors, and one with spoken captions as anchors.
It can be written as:
\begin{align}
    \label{retrieval_triplet}
    T(I_j, C_j, I_j^{imp}, C_j^{imp}) &= \sum_{j=1}^N \Big(\max(0,S(I_j, C_j^{imp}) - S(I_j, C_j) +1) \\
                                      &{\quad\quad\quad+} \max(0,S(I_j^{imp}, C_j) - S(I_j, C_j) +1)\Big)
\end{align}
Where $S$ is a chosen similarity function,  $I_J$ and $C_j$ are the $j\th$  image/caption which are treated as , and $I_j^{imp}$ and $C_j^{imp}$ are the $j\th$ \textit{imposter} samples of an image and caption not associted with the correct image/caption pair.
In this formulation, the first $\max$ term is effectively the triplet loss when  $I_j$ is viewed as the anchor, $C_j$ is viewed as $T$, and  $C_j^{imp}$ is viewed as $F$.
The second $max$ term then uses $C_j$ as the anchor and the images in likewise fashion.
The imposters are sampled from a uniform distribution from within the minibatch.

This form of triplet loss is highly dependent on the imposter samples chosen, since in the final stages of optimization most negative samples will alredy be far from the anchor.
This motivates hard and semi-hard negative sampling \cite{jansen2018unsupervised} which, generally speaking, seek to sample imposter/negative samples that are very close to the anchor (and there for `hard' to distiguish from the positives).
This general approach was adapted to the spoken image retriaval task by Harwath et al. \cite{harwath2018jointly} which showed that the use of semi-hard negatives increased retriaval performance.

The accompanying loss function, was formulated more generally in \cite{harwath2019learning} as:
\begin{align}
    \label{eq:triplet_full}
    \mathcal{L}(\theta) &= T(I_j, C_j, I_j^{imp}, C_j^{imp}) + T((I_j, C_j, \tilde{I}_j^{imp}, \tilde{C}_j^{imp})
\end{align}
Where $T$ is the same as in Equation \ref{retrieval_triplet}. 
The important difference between the terms is that $\tilde{I}_j^{imp}$ and $\tilde{C}_j^{imp}$ are chosen to be the image and caption that are most similar to their respective anchors in the batch (as measured by the similarity function $S$),
instead of the uniform distribution as $I_j^{imp}$ and $C_j^{imp}$


\subsection{Masked Margin Softmax (MMS)}
\label{section:mms_loss}
The Masked Margin Softmax (MMS) loss was introduced by Ilharco et al. \cite{ilharco2019large} and uses many more negative examples than triplet loss.
MSS loss effectively computes the categorical cross-entropy loss between an anchor embedding and $K$ other embeddings, only one of which is the positive corresponding embedding.
Letting $B$ be the batch size and $X$ and $Y$ be the final representations of $B$ samples of two arbitrary informations sources.
We use information souce to refer to one of the followinge: an image, or a speech utterance from a language in the set of $\mathcal{D}_{lang}$ languages in the dataset. 
MMS can then be written for a single minibatch as:
\begin{align}
    \label{MMS_base}
    MMS(X,Y) = \frac{1}{B}\sum_{j=1}^B\frac{e^{S(X_j, Y_j)-M}}{e^{S(X_j, Y_j)-M} +\sum_{k=1}^{B} \mathbb{I}[k!=j \land k \notin \mathcal{Z}]e^{S(X_j, Y_k)}}
\end{align}
Where $\mathcal{Z}$ is a set of predefined indices corresponding to examples to be \textit{masked} (hence the name).
In our dataset there are no examples of this sort so $\mathcal{Z}$ is always empty.

Notice that $MMS(X, Y)$ is not symmetric. 
Specifically, each representation $X_j$ is contrasted with all other $Y_k$ in the batch, by $Y_j$ is not given the same treatment.
The full MMS loss for a batch completes the symmetry and is defined as:
\begin{align}
    \label{eq:mms}
    \mathcal{L}_{MMS}(\theta) = MMS(X,Y) + MMS(Y,X)
\end{align}

The margin $M$ is an important quantity and can be a constant value, updated according to a schedule, or adaptively updated according to some function of the batch.
When $M=0$ this loss has the form of what is often referred to in the literature as InfoNCE loss \cite{oord2018representation}.
Several works use this terminology \cite{he2020momentum, hjelm2018learning}, but there seems to be a lack of consensus over this issue \cite{tian2020contrastive}.
Still, we hesitantly follow the current conventions and whenever an experiment uses a margin of 0, we will refer to it as InfoNCE.

These methods are explored empirically in Section \ref{section:obj_exp_testing_rep}.

\subsection{Hypersphere Loss}
The hypersphere loss was introduced by Wang \cite{wang2020understanding} and attempts to reframe conventional contrastive loss through the notions of \textit{alignment} and \textit{uniformity} of latent representaions on the hypersphere.
They provide empirical evidence that directy optimizing their measures of alignment and uniformity can outperform contrastive losses.
Given a batch of size $B$ with $X$ and $Y$ defined in the same manner as in Section \ref{section:mms_loss}, the \textit{alignment} sub-loss is defined as:
\begin{align*}
    HSphere_{align}(X,Y) = \frac{1}{B}\sum_{i=1}^B \left\Vert X_i - Y_i\right\Vert_2^\alpha
\end{align*}
where $\alpha$ is a hyperparameter to be tuned.

The \textit{uniformity} sub-loss is calculated as:
\begin{align*}
    HSphere_{uniformity}(X,Y) = \frac{1}{B(B+1)/2}\sum_{i=1}^B\sum_{j\geq i}^B e^{-t\left\Vert X_i - Y_j\right\Vert_2^2}
\end{align*}
with $t$ being another hyperparameter.

In the original work the showed that values of $\alpha$ and $t$ around 1 tended to work well, but $t$ often needed to be lower than $alpha$.

With those defined the full hyperspheric loss is defined as:

\begin{align}
    \label{eq:hyperspheric}
    HSphere_{full}(X,Y) = \lambda_a*HSphere_{align}(X,Y) + \lambda_u*HSphere_{uniformity}(X,Y)
\end{align}
With $\lambda_a$ and $\lambda_u$ being additional weiting parameters.
In the original work, an additional baseline contrastive loss was also added to Equation \ref{eq:hyperspheric} with an accompanying weighting hyperparameter, but they found that it was not necessary for optimal performance so we have omitted it in our experiments for simplicity.


\section{Multiview Contrasting Frameworks}
\label{section:multiview_framework}
Work by Tian et al. \cite{tian2020contrastive}, provided a framework for using contrastive loss functions with several \textit{views}.
They define a view as a certain sensory input during an arbitrary event.  
In this way, there may be multiple views all describing the same underlying (or latent) event.
A view, then, can have a more abstract meaning in which some amount of information contained in the view is shared among other views.

Their work, then attempts to devise appropriate training schemes to maximize the latent information shared by all views.
The two components of their work relevant to ours are the \textit{full-graph} and \textit{anchor} training schemes.

Both schemes start with a symmetric loss function, $\mathcal{L}_{sym}(V_1,V_2)$, defined over two views, $V_1$ and $V_2$, of size $B$.
Each samples in $V_1$ are assumed to have a corresponding positive sample in $V_2$ that carries the same mutual information (and vice-versa for the samples in $V_2$).
Note that $V_1$ and $V_2$ are from a minibatch of the full dataset, but to reduce notational clutter we will only define a single batch update.
$\mathcal{L_{sym}}$ is also assumed to be designed such that it encourages alignment between the positive examples in two views and some notion of distance between the examples that do not carry mutual information.
We will assume as well that the positive examples share the same index in the batch. 
Notice that Equations \ref{eq:hyperspheric}, \ref{eq:triplet_full}, and \ref{eq:mms} all meet the above assumptions.

Finally, the full-graph and anchor frameworks attempt to maximize the mutual information contained in a set of $K$ views $\mathcal{V_i}, i\in [1,K]$.

% \subsection{Full Graph}
% \label{section:full_graph_framework}
With the above assumptions and notation, the full graph framework computes the following loss for each batch:
\begin{align*}
    \mathcal{L}_{full_g}(\theta) = \frac{1}{B(B+1)/2}\sum_{i=1}^B \sum_{j\geq k}^B \mathcal{L}_{sym}(V_i, V_j)
\end{align*}

The anchor framework can be defined intuitively as aligning all views to one `anchor' view in a hub-and-spoke manner.
It is calculated as:
\begin{align*}
    \mathcal{L}_{anchor}(\theta, i) = \frac{1}{B-1}\sum_{j=1}^{B} \mathbb{I}[j\neq i] \mathcal{L}_{sym}(V_i, V_j)
\end{align*}
Where $i$ denotes the index of the anchor view.
This formulation clearly makes every contrasting equation involve $\mathcal{V}_i$ making it `anchor' all other views.
% \subsection{Anchor}
% \label{section:anchor_framework}
\section{Neural Architectures}
\label{background:model_architectures}
We use convolutional neural networks (CNN) as our base architecture. 
The focus of this work is on the effect of loss functions and training regime in a multilingual image retrieval and as such does not attempt to optimize architecture.
We therefore use two off the shelf encoders for our images and speech utterances.
\subsection{Image Model}
The image encoder is based on the ResNet50 model proposed by He et al. \cite{he2016deep}.
It is pretrained on ImageNet classification, but the final layer has been altered slightly.
In the same manner as Harwath et al. \cite{harwath2019learning}, we remove the final softmax and fully connected layer and instead perform a 1x1 convolution to obtain a desired size of the final imagage representations.
The output representation size for each image is a 7x7x1024 tensor.
\subsection{Audio Model}
\label{chapter:background|section:audio_model}
We will first describe the pre-processing steps taken on the audio and then briefly describe the architecture.
\subsubsection{Audio Pre-Processing}
The same pre-processing is used for all experiments.
Log-Mel filter bank spectrogram features are calculated from raw audio with 40 frequency bins.
We use Hamming-windowed frames with a 15 ms width and shift of 10 ms.
During batching, all utterances are padded up to the longest utterance in the batch.
\subsubsection{Architecture}
The audio/speech model we use the same as \cite{harwath2019learning}. 
It is a 17 layer CNN with the first being a 1D convolution of 128 filters of size 1x40x1 across the time dimension.
The following 16 layers divided into 4 residual \textit{speech} blocks.
Each speech block starts with an initial 1D convolution layer with kernel size of 1x9 a stride of two that effectively downsamples the input followed by a batch norm layer.
This followed by three successive 1D convolution (kernel size 1x9, stride 1) and batch norm layers.
No padding is used.
This produces 1024 dimensional embeddings of various length depending on the batch.



% \section{Multi-lingual Training Frameworks}
% \label{section:multiling_framework}
%
% \subsection{Language Tag}
% \subsection{Shared Layers}


\chapter{Exploring Learining Objectives}
\index{Exploring Learining Objectives}
\label{chapter:objective_exploration}

This chapter explores the extent to which the overall learning objective effects retrieval performance, indicating better alignment with respect to a similarity metric.
We note that the terms `learning objective' and `loss function' are used in this chapter interchangeably.
% However, in certain experiments in this chapter additional parameters are added to the prediction pipeline for a particular learning objective, which is atypical 




\section{Experimental Design}
This section describes the procedures used for out data preparation, training, and evaluation.
We also describe variations to our model's output that we will explore.
The base architecture, though, is unchanged from the description in Section \ref{background:model_architectures}.
\label{chapter:obj_exp|section:exp_design}
\subsection{Data}
The experiments in this chapter (and all proceeding chapters) use a dataset consisting of 100,000 images from the Places 205 dataset \cite{zhou2014learning} that have additional English, Hindi \cite{harwath2018interlingua}, and Japanese \cite{ohishi2020trilingual} spoken descriptions corresponding to each image.
It is important to note that the descriptions from each language are not translations of each other as each speaker is only directed to describe the image and has no knowledge of what was said previously. 
This means that there is no assumption of alignment in the ordering of objects described or the way in the manner in which the speaker chooses to describe the image.
\dftwrds[TODO: make this] The statistics for the descriptions can be found in Figure \dftwrds.

We use a training set of 99,000 datapoints and use the remaining 1,000 datapoints for testing. 
% In cases in which hyperparameter tuning is performed, such as Section a random subset of 1,000 datapoints are extracted from the training set and used for evaluation.
% \subsection{Models}
% \label{section:obj_exp_models}
% The experiments in this chapter all use the same basic implementation for the full encoder function $f_\theta$.
% All viewpoints of the image view are encoded using an image-specific encoder.
% Additionally, each language view has it's own dedicated speech encoder, meaning there are no shared layers/parameters across languages.
% However, the architecture of each language's encoder is identical.
%
% All models used in this work are convolutional neural networks (CNNs) at their core.
% The details of each CNN architecture are given below.
%
% \subsubsection{Image Encoder}
% The image encoder we use is a resnet CNN published by \citeme[He et al] and is initialized using the weights of a model that has been pre-trained on imagenet \citeme{} object classification.
% The image encoder has \dftwrds{architecture specs here}.
% For our dataset the representation of each image output by the image encoder has fixed dimensions and is in $\mathbb{R}^{7\times7\times1024}$.
% Where 1024 is out embedding size of each super pixel.
%
% % This model has been pre-trained on the task of object detection by using the imagenet dataset \citeme{}d
%
% \subsubsection{Audio/Speech Encoder}
% \label{section:speech_encoder}
% The audio encoder architecture is the same across all languages.
% It is a resnet CNN introduced by \citeme[Harwath resnet paper].
% The architecture is similar to the image encoder, but with slight differences in the dimensions in each layer.
% These differences are a natural result of the inherent differences in the visual and auditory modalities.
% % presumably a results of the original authors' optimization efforts on the respective modalities.
% The details of the architectural differences are given in figure \dftwrds[make and reference figure].
% All audio encoders are randomly initialized.
%
% Unlike the image encoder, since spoken utterance is of variable length, the dimensions of the speech output for each encoder are not fixed.
% To remedy this, for each batch we take each language view and find the longest utterance in the batch.
% We then pad all other utterances in the respective view to be the same length as the longest.
% This means the sequence length of each language view is fixed inside a batch, but the sequence lengths vary across languages within a batch and across batches.
% Thus, letting $i$ denote the $i\th$ batch and $j$ denote the $j\th$ view, we can define $max\_sl_{ij}$ as the maximum sequence length of the utterance of the $j\th$ language view in the $i\th$ batch.
% Due to the nature of the speech encoder's CNN architecture, the sequence length of the output representation has a reduced length, but is deterministic and we can denote it $max\_sl\_out_{ij}$ for simplicity.
% For our dataset, $max\_sl\_out_{ij}$ typically ranges from 50 to 500.
% With this definition we have that the output of each speech encoder is in $\mathbb{R}^{max\_sl\_out_{ij}\times 1024}$.
% Where 1024 is out embedding size of the output sequence.
%
\subsection{Training}
\label{section:obj_exp_training}
% In this chapter, all views are encoded with their own view specific model.
We use the Adam optimization algorithm in all of the following experiments with a momentum setting of .99 \dftwrds[check this].
We also use a batch size of 128 and a learning rate of .001 with a scheduler. 
This scheduler enforces a linear warmup schedule (from 0 to .001) through the first 10\% of training steps, and then decreases by a factor of .99 after every 50 steps.

In this chapter's experiments, all training is done using the full-graph framework of multi-view training as described in Section \ref{section:multiview_framework}.
This is to simplify experimental variables, such as choosing which view will be the anchor, as well as to give easy access to all information shared between the views.
The next chapter will explore ways to reduce the complexity of full pairwise comparisons.

Importantly, all loss functions under consideration have some component designed to encourage dissimilitude. 
Without this, predictions can find a trivial local solution by collapsing on constant value for all representations.
Contrastive loss functions use `negative examples' to serve this role where as loss functions like the hyperspheric loss contain the uniformity objective.
In either case, we supply the non-positive examples in each batch to these sub-components. 

To be clear, all loss functions draw their negative examples from within their respective batch.
%InfoNCE, Triplet loss, Masked Margin loss, and Scheduled Masked Margin loss all 
% The Hyperspheric loss additionally draws the samples for it's uniformity loss term from the examples in the batch.
% This strategy is in contrast to updating a memory-bank mechanism \citeme in which a much larger number of stale representations are used and periodically updated.


\subsection{Testing and Reporting Metrics}
\label{section:obj_exp_testing_rep}
During inference, for each datapoint, each viewpoint is measured with respect to a similarity function with every viewpoint in other views. 
We use the dot product as our similarity function.
Recall at 1, 5, and 10 are then calculated by checking if the correct target viewpoint ranked in the top 1, 5, or 10 most similar.
% To clarify, if $N$ be the
% \dftwrds[fix the previous sentence. Merge with explanation below]
% We report the recall at k scores with $k=1, 5,$ and $10$, which indicates how often the target viewpoint was in the top $k$ closest viewpoints.
Also, because of space concerns and since are primarily interested in overall alignment in the latent space, for each view pair we report only the average recall at k scores between view pairs.
That is, for each view pair, two sets of recall at k scores are calculated: 1) a set in which the first view of the pair is treated as the input and the second view is the target, and 2) a set where the second of the pair is treated as the input and the first view is the target.
These two directions do not necessarily produce the same results, but we did not observe a systematic of substantial deviation from the average among any view pair in any experiment in this work.
We therefore chose to omit the scores for the individual directions and simply report the average.

Still, we fundamentally distinguish between view pairs between different modalities (i.e. image-speech pairings) and those withing the same modality (speech-speech).
For convenience, then, we will refer to average results from speech-speech retrieval scores simply as `cross-lingual retrieval results' and average results from speech-image pairings simply as `image retrieval results', acknowledging that retrieving speech utterances from image input is not actually `image retrieval'.
\subsection{Experimental Variables}
We separately  explore two sets of experimental variables in this chapter: loss functions and output pooling mechanisms.
\subsection{Loss Functions}
The loss functions under consideration were described in detail in Section \ref{chapter:background|section:loss_functions}, but we will explicitly compare:
\begin{enumerate}
    \item The hyperspheric loss (Equation \ref{eq:hyperspheric})
    \item The triplet loss (Equation \ref{eq:triplet_full})
    \item InfoNCE (Equation \ref{eq:mms} with $M=0$)
    \item Masked Margin Softmax, M=1 (Equation \ref{eq:mms})
    \item Masked Margin Softmax, scheduled updates for M following the original work by Ilharco et al. \cite{ilharco2019large}
    \item Masked Margin Softmax, with adaptive updates as introduced by Monfort et al. \cite{monfort2021spoken}

\end{enumerate}
\subsection{Output Pooling}
\label{section:output_pooling}
As noted in the previous section, the output of each view encoder do not necessarily align across all dimensions. 
This creates a problem for most distance/similarity measures which typically require vector representations.
Work by \cite{harwath2018jointly} explored a more complex set of similarity measures which utilize similarity measures across the time and location dimension, but for this work we restrict ourselves to a simplified setting of similarity measures defined only on two vectors of the same dimension.
This is also more realistic in terms of retrieval when the set of target viewpoints is very large.
We therefore require some sort of pooling mechanism before similarity scores can be calculated and retrieval can be performed.


Experiments in section \ref{section:pooling_experiments} compare pooling strategy performance empirically, but the following sections describe the basics of each mechanism.

\subsubsection{Average Pooling}
\label{section:average_pooling}
This is the simplest pooling strategy and consists of first flattening all dimensions before the last (only relevant for the image view), then averaging across the flattened dimension.
One complication is the padding added to resolve the variability of sequence length in the speech encoders.
Using a simple average would diminish the features of the meaningful parts of shorter sequences.
Also, the sequence length is a reduced due the convolutional downsampling in each layer, which means there may be a point in the final output sequence that corresponds to both padded and unpadded input.
To deal with this we only average the first $avg_pool_len$ sequence embeddings. 
For the $k\th$ utterance of the  $i\th$ batch of the $j\th$ view, we denote $len_{ijk}$ as the original sequence length of the utterance. 
we calculate $avg\_pool\_len$ as:
\begin{equation*}
    avg\_pool\_len = \lfloor  len_{ijk} / \text{round}\Big( \frac{max\_sl_{ij}}{max\_sl\_out{ij}} \Big) \rfloor
\end{equation*}
Where $max\_sl_{ij}$ and $max\_sl\_out_{ij}$ are defined in the same manner as in section \ref{chapter:background|section:audio_model} and $\text{round}(\cdots)$ rounds to the nearest integer.
Here $\frac{max\_sl_{ij}}{max\_sl\_out{ij}}$ can be thought of as the reduction ratio from the input sequence length to the output sequence length.
% This calculation follows the same strategy as in \citeme[Harwath's resnet orig or VQ paper].

\subsubsection{Multi-Head Self-Attention Pooling}
\label{section:mh_attn_pooling}
Considering the simplistic nature of average pooling, we also experiment with using a self attention layer to perform the pooling.
Previous work \cite{ohishi2020trilingual} found that a single headed self-attention layer was beneficial when used \textit{before} the pooling layer (specifically between the second to last and last convolutional layers), but they still used average pooling for their final representation.

Inspired by this result and the recent proliferation and success of Transformer layers, which uses multi-headed self-attention (MHSA), we attempt to use this as an adaptive pooling mechanism for our final representation.
Specifically, we use 8 heads and an positional encoder layer and we chose the larges $max_sl_out_ij$ to be the maximum sequence length for the positional encoder.
However, unlike the original transformer block, we do not scale the input by $\sqrt{d_{model}}$ and instead scale the positional encoding by $\frac{1}{\sqrt{d_{model}}}$  as we found this produced better results.
We hypothesize that without the layer norm used after the MHSA layer in the transformer block, the original scaling gives final representations an overly large L2 norm which negatively effects our similarity based losses.
We also note, though, that adding a layer norm layer and a residual connection as in the original transformer block did not resolve this issue.
Further inquiry would be required to fully explain this behavior.

After the positional encoding, we apply a dropout layer to impose a layer specific hyperparameter we can use to counteract the potential for MHSA to overfit to the dataset.

Another implementation note is that we chose to prepend an additional learnable embedding to each flattened/sequence dimension to act in the same capacity as the CLS token used in the transformer block.
We then use the first embedding (which corresponds to the position of prepended embedding) as our final representation.
We also experimented with simply using the first embedding of the sequence and the results of this are discussed in section \ref{section:pooling_experiments}
% but found that this also negatively impacted results (shown in Section \ref{section:pooling_experiments}.
% For our main experiments chose to use the additional embedding since it better aligns with implementation used in the Transformer.

As a final note, we use sequence masking to prevent the final representation from attending to the embeddings corresponding to padding.
And finally, we used the PyTorch implementation of MHSA in our experiments.

\subsubsection{Transformer Pooling}
\label{section:transformer_pooling}
We also experiment with using a full Transformer block and using a single embedding to serve as the final representation. 
We used the same settings and adaptations as described in \ref{section:mh_attn_pooling} regarding number of heads, positional encoding, prepending a learned embedding, and masking.
We chose to use 2048 as the internal feed forward dimension, which is the same as the original Transformer.


\subsection{Loss Functions Experiments}
This chapter's experiments involve distinct loss functions: InfoNCE, hyperspheric loss, triplet loss, masked margin loss, and scheduled mml. 
We compare performance of each directly in section \ref{section:direct_compare}.
However, the hyperspheric loss contains many hyperparameters that are unique to it.
Also, this loss function has never been applied to this task so optimal hyperparameters have not previously been studied.
We therefore describe our tuning procedure in section \ref{section:hyperspheric_tuning}.

After comparing loss functions, we also experiment with different pooling strategies in section \ref{section:pooling_experiments} and additional potential settings for loss functions in section \ref{section:addition_settings}.




\subsection{Hyperspheric Loss Experiments}
\label{section:hyperspheric_tuning}
As the hyperspheric loss is markedly different than the loss functions previously explored on this dataset, we chose to tune the uniformity sub-loss weighting, $\lambda_{unif}$.
We chose this hyperparameter because it appeared to be the most impactful to the final results of the original paper \cite{wang2020understanding}.
There are in fact several hyperparameters available for adjustment for the hyperspheric loss, but due to time constraints we were unable to tune the others.
We chose an optimization scheme of staring with the baseline value (1.0), and decreasing by $.25$ until performance ceased to increase. 
We kept all other hyperparameters constant during this tuning. 
In particular, we held the alignment loss weighting, $\lambda_{align}$ at 1.0, $t=2$, and $\alpha=2$.

We used the encoder implementations described in \ref{background:model_architectures} and used average pooling to combine the final output representations



\begin{table}
    \centering
    \begin{tabular}{l|l|l|l|l}
        \toprule
        {} & Unif. 1 & Unif .875 & Unif .75 & Unif. 50 \\
        \midrule
        Eng.\&Img.r1  &            0.05\% &                 0.00\% &                8.30\% &                0.05\% \\
        Eng.\&Img.r5  &            0.55\% &                 0.50\% &               25.20\% &                0.35\% \\
        Eng.\&Img.r10 &            1.40\% &                 1.05\% &               36.95\% &                0.90\% \\
        \midrule
        Hin.\&Img.r1  &            0.10\% &                 0.10\% &                7.85\% &                0.10\% \\
        Hin.\&Img.r5  &            0.45\% &                 0.50\% &               23.10\% &                0.50\% \\
        Hin.\&Img.r10 &            0.70\% &                 1.25\% &               33.75\% &                0.90\% \\
        \midrule
        Jap.\&Img.r1  &            0.10\% &                 0.00\% &                9.00\% &                0.05\% \\
        Jap.\&Img.r5  &            0.80\% &                 0.40\% &               30.55\% &                0.40\% \\
        Jap.\&Img.r10 &            1.40\% &                 1.10\% &               43.65\% &                0.95\% \\
\end{tabular}


\caption{Hyperspheric loss  w/ Average pooling}
\label{table:hyper_img_ret}
\end{table}


\begin{table}
    \centering
    \begin{tabular}{l|l|l|l|l}
        \toprule
        {} & Unif. 1 & Unif .875 & Unif .75 & Unif. 50 \\
        \midrule
        Eng.\&Hin.\_r1  &            0.20\% &                 0.25\% &                5.90\% &                0.05\% \\
        Eng.\&Hin.\_r5  &            1.25\% &                 1.20\% &               18.00\% &                0.50\% \\
        Eng.\&Hin.\_r10 &            2.00\% &                 3.00\% &               27.05\% &                1.00\% \\
        \midrule
        Eng.\&Jap.\_r1  &            0.00\% &                 0.15\% &                6.90\% &                0.05\% \\
        Eng.\&Jap.\_r5  &            0.20\% &                 0.95\% &               21.30\% &                0.30\% \\
        Eng.\&Jap.\_r10 &            0.55\% &                 1.60\% &               33.95\% &                0.95\% \\
        \midrule
        Jap.\&Hin.\_r1  &            0.20\% &                 0.10\% &                6.40\% &                0.15\% \\
        Jap.\&Hin.\_r5  &            0.60\% &                 0.30\% &               19.30\% &                0.60\% \\
        Jap.\&Hin.\_r10 &            0.90\% &                 0.55\% &               28.95\% &                1.35\% \\
        \bottomrule
\end{tabular}


\caption{Hyperspheric loss  w/ Average pooling}
\label{table:hyper_cling}
\end{table}


The results for the image-audio (image retrieval) view pairings can be found in figure \ref{table:hyper_img_ret} and the audio-audio pairs (aka cross-lingual) shown in figure \ref{table:hyper_cling}.
As can be seen in both image retrieval and cross-lingual settings, the overall performance of the hypersheric loss is highly dependent on the weighting of the uniformity sub-loss.
In all following experiments, we use the best performing uniformity weighting, .75.

\subsection{Loss Function Experiments}
\label{section:direct_compare}

This set of experiments compares the results of several prominent loss functions that been applied to image retrieval tasks.
Specifically, we compare InfoNCE loss, triplet loss, masked margin softmax loss, adaptive mean margin softmax loss, and hyperspheric loss with the hyperparameters described in section \ref{section:hyperspheric_tuning}.
Training, inference, and evaluation procedures are as described in sections \ref{section:obj_exp_training} and \ref{section:obj_exp_testing_rep}.
% All losses are compared with the same encoder function $f_\theta$ as described in section \ref{section:obj_exp_models}.
% nd pooling strategy.
% Specifically, these experiments use a resnet50 image encoder model described in section \dftwrds[ref later] and separate 


\begin{table}
    \centering
\begin{tabular}{l|llllll}
\toprule
{} & Hyper. & Triplet & InfoNCE & MMS & MMS & MMS \\
{} & Un=.75 &         &         & M=1 & Sched. & Adapt \\
\midrule
Best Epoch  &                   16 &                        36 &                      26 &            19 &                  24 &                  29 \\
\midrule
Image Retrieval &&&&&&\\
\midrule
\midrule
E\&I.avgR1   &                2.90\% &                    16.40\% &                  18.05\% &        12.65\% &              18.85\% &              16.90\% \\
H\&I.avgR1   &                3.15\% &                    12.95\% &                  14.70\% &        16.15\% &              16.50\% &              14.45\% \\
J\&I.avgR1   &                3.70\% &                    22.70\% &                  28.00\% &        25.50\% &              28.35\% &              27.30\% \\
\midrule
E\&I.avgR5   &               12.65\% &                    39.20\% &                  41.55\% &        36.65\% &              42.15\% &              45.25\% \\
H\&I.avgR5   &               12.10\% &                    32.65\% &                  36.10\% &        36.45\% &              37.65\% &              36.90\% \\
J\&I.avgR5   &               15.20\% &                    53.15\% &                  59.20\% &        57.45\% &              58.05\% &              61.05\% \\
\midrule
E\&I.avgR10  &               22.55\% &                    51.55\% &                  54.55\% &        50.30\% &              54.90\% &              57.50\% \\
H\&I.avgR10  &               20.95\% &                    43.55\% &                  45.85\% &        47.60\% &              46.85\% &              47.15\% \\
J\&I.avgR10  &               26.70\% &                    67.05\% &                  72.35\% &        71.50\% &              71.95\% &              74.40\% \\
\midrule
Cross-Ling &&&&&&\\
\midrule
\midrule
H\&E.avgR1   &                1.75\% &                     9.65\% &                  11.55\% &         8.35\% &              12.00\% &              10.65\% \\
J\&E.avgR1   &                1.40\% &                     8.05\% &                  11.40\% &         7.90\% &              11.25\% &              12.25\% \\
J\&H.avgR1   &                2.05\% &                     6.30\% &                  10.50\% &         8.85\% &               9.40\% &              10.15\% \\
\midrule
H\&E.avgR5   &                8.10\% &                    23.35\% &                  27.25\% &        22.10\% &              28.45\% &              26.30\% \\
J\&E.avgR5   &               10.75\% &                    24.70\% &                  29.00\% &        25.00\% &              32.25\% &              32.00\% \\
J\&H.avgR5   &                8.95\% &                    21.80\% &                  27.30\% &        24.70\% &              26.35\% &              25.40\% \\
\midrule
H\&E.avgR10  &               13.95\% &                    32.85\% &                  37.55\% &        31.80\% &              38.00\% &              36.00\% \\
J\&E.avgR10  &               20.00\% &                    35.85\% &                  41.30\% &        36.85\% &              44.60\% &              42.40\% \\
J\&H.avgR10  &               15.80\% &                    30.70\% &                  36.80\% &        35.60\% &              36.60\% &              35.70\% \\
\bottomrule
\end{tabular}


% \begin{tabular}{llll}
% \toprule
% {} & Hyper. & InfoNCE & Triplet \\
% \midrule
% Best Epoch          &                   16 &                      26 &                        36 \\
% Jap.\&Eng.\_r1  &                6.90\% &                  11.40\% &                     8.05\% \\
% Jap.\&Eng.\_r5  &               21.30\% &                  29.00\% &                    24.70\% \\
% Jap.\&Eng.\_r10 &               33.95\% &                  41.30\% &                    35.85\% \\
% \midrule
% Hin.\&Eng.\_r1  &                5.90\% &                  11.55\% &                     9.65\% \\
% Hin.\&Eng.\_r5  &                18.00\% &                  27.25\% &                    23.35\% \\
% Hin.\&Eng.\_r10 &               27.05\% &                  37.55\% &                    32.85\% \\
% \midrule
% Jap.\&Hin.\_r1  &                6.40\%  &                  10.50\% &                     6.30\% \\
% Jap.\&Hin.\_r5  &               19.30\%  &                  27.30\% &                    21.80\% \\
% Jap.\&Hin.\_r10 &               28.95\% &                  36.80\% &                    30.70\% \\
% \bottomrule



\caption{Objective comparison  w/ average pooling on cross-lingual retrieval}
\label{table:Objective_results}
\end{table}

% 
\begin{table}
    \centering
\begin{tabular}{llll}
\toprule
{} & Hyper. & InfoNCE & Triplet \\
\midrule
Best Epoch          &                   16 &                      26 &                        36 \\
Jap.\&Eng.\_r1  &                6.90\% &                  11.40\% &                     8.05\% \\
Jap.\&Eng.\_r5  &               21.30\% &                  29.00\% &                    24.70\% \\
Jap.\&Eng.\_r10 &               33.95\% &                  41.30\% &                    35.85\% \\
\midrule
Hin.\&Eng.\_r1  &                5.90\% &                  11.55\% &                     9.65\% \\
Hin.\&Eng.\_r5  &                18.00\% &                  27.25\% &                    23.35\% \\
Hin.\&Eng.\_r10 &               27.05\% &                  37.55\% &                    32.85\% \\
\midrule
Jap.\&Hin.\_r1  &                6.40\%  &                  10.50\% &                     6.30\% \\
Jap.\&Hin.\_r5  &               19.30\%  &                  27.30\% &                    21.80\% \\
Jap.\&Hin.\_r10 &               28.95\% &                  36.80\% &                    30.70\% \\
\bottomrule
\end{tabular}

\caption{Objective comparison  w/ average pooling on cross-lingual retrieval}
\label{table:Objective_cling_ret}
\end{table}
l

Results for image retrieval pairings and cross-lingual results can be found in Table \ref{table:Objective_results}.
We highlight several findings from this experiment:
\begin{enumerate}
    \item InfoNCE loss consistently outperforms all other losses in both image retrieval and cross-lingual settings.
    \item Cross-lingual retrieval appears more difficult than image retrieval for all losses.
    \item Japanese and image representations appear remarkably strongly aligned for both InfoNCE and triplet loss.
\end{enumerate}

Finding 1
% generally aligns with the results reported by \cite{ohishi2020trilingual}.
% This experiment, however, 
provides empirical evidence that InfoNCE outperforms a broad range of loss functions.
Experiments by \cite{ohishi2020trilingual} showed that MMSM outperformed the form of triplet loss in our experiments, but our finding provide evidence that InfoNCE outperforms a broader range of loss functions.


Finding 2 and 3 also aligns with the general findings of \cite{ohishi2020trilingual}, and our experiments show that this trend continues for InfoNCE.
Considering the strong performance, all remaining experiments will use InfoNCE loss.

\subsection{Pooling Experiments}
\label{section:pooling_experiments}


\begin{table}
    \centering
\begin{tabular}{lllll}
\toprule
{} & Lr .001 & Lr .0001 & Lr .0001 & Lr .0001 \\
{} & Drop .1 & Drop .1 & Drop .3 & Drop .5 \\
\midrule
Best Epoch          &                          1 &                                  15 &                     21 &                     32 \\
\midrule
eng\&img\_r1  &                      0.25\% &                              14.10\% &                 11.05\% &                  4.55\% \\
eng\&img\_r5  &                      0.70\% &                              34.20\% &                 33.80\% &                 14.15\% \\
eng\&img\_r10 &                      1.55\% &                              45.70\% &                 44.85\% &                 23.50\% \\
\midrule
hin\&img\_r1  &                      0.05\% &                               9.70\% &                  9.05\% &                  3.65\% \\
hin\&img\_r5  &                      1.15\% &                              27.40\% &                 25.35\% &                 13.60\% \\
hin\&img\_r10 &                      1.75\% &                              37.70\% &                 34.65\% &                 21.25\% \\
\midrule
jap\&img\_r1  &                      0.15\% &                              18.70\% &                 19.10\% &                  6.20\% \\
jap\&img\_r5  &                      1.10\% &                              46.85\% &                 45.35\% &                 20.85\% \\
jap\&img\_r10 &                      2.10\% &                              62.15\% &                 59.45\% &                 30.05\% \\

\bottomrule
\end{tabular}

\caption{Multi-head attention pooling using InfoNCE loss. Image retrieval results}
\label{table:mh_attn_hparams_img_ret}
\end{table}

\begin{table}
    \centering
\begin{tabular}{lllll}
\toprule
{} & Lr .001 & Lr .0001 & Lr .0001 & Lr .0001 \\
{} & Drop .1 & Drop .1 & Drop .3 & Drop .5 \\
\midrule
Best Epoch          &                          1 &                                  15 &                     21 &                     32 \\
\midrule
hin\&eng\_r1  &                      0.20\% &                               7.60\% &                  6.50\% &                  2.20\% \\
hin\&eng\_r5  &                      1.65\% &                              20.85\% &                 18.80\% &                  9.25\% \\
hin\&eng\_r10 &                      2.70\% &                              29.20\% &                 27.80\% &                 14.80\% \\
\midrule
jap\&eng\_r1  &                      0.15\% &                               6.70\% &                  6.25\% &                  2.05\% \\
jap\&eng\_r5  &                      0.70\% &                              19.80\% &                 19.05\% &                  7.95\% \\
jap\&eng\_r10 &                      1.05\% &                              29.30\% &                 26.90\% &                 12.90\% \\
\midrule
jap\&hin\_r1  &                      0.20\% &                               5.75\% &                  5.00\% &                  1.85\% \\
jap\&hin\_r5  &                      0.80\% &                              17.15\% &                 15.65\% &                  7.85\% \\
jap\&hin\_r10 &                      1.45\% &                              26.50\% &                 22.40\% &                 12.60\% \\

\bottomrule
\end{tabular}

\caption{Multi-head attention pooling using InfoNCE loss. Cross-lingual retieval results.}
\label{table:mh_attn_hparams_cling}
\end{table}

As described in \ref{section:output_pooling}, there are many different strategies to pool the final output representations of each view.
Since neither MHSA or a Transformer layer has been used a pooling strategy, we first attempt to tune hyperparameters of the simpler of the two layers, the MHSA.
We chose to adjust the learning rate and the dropout percentage.
Due to time constraints we were not able to fully tune these parameters, but we were able to explore an array of settings. 
The image retrieval scores for these settings can be found in Table \ref{table:mh_attn_hparams_img_ret} and the cross-lingual scores can be found in Table \ref{table:mh_attn_hparams_img_ret}.

Comparing these results with those for average-pooled InfoNCE, MHSA is clearly underperforming.
We first found that the original learning rate of .001 made learning unstable and decreasing to .0001 helped considerably.
We next hypothesized that the lower performance might be due to some amount of overfitting of the attention mechanism.
This prompted us to adjust the percentage of dropout applied after the positional encoding layer.
Our results contradict this hypothesis as it seems increasing regularization did not improve generalization.




\subsubsection{Additional settings}
\label{section:addition_settings}
These above results prompted us to try several other uses of MHSA as well as the Transformer block.
First, we tried to simply use the first embedding of the flattened sequence instead of the additional learned prepended token.
We also tried to place the MHSA between the second to last and last layer and simply feed that to the last convolutional layer to then get average-pooled.
This is a similar strategy of that explored by \cite{ohishi2020trilingual}, but we maintain our 8 head configuration rather than using a single head.
The learning rate for this setting was returned to .001 (since that worked well previously for average pooling) and dropout the dropout kept at .1.
Finally we try the Transformer block, which has an additional layer norm layer and two linear layers.
\begin{table}
    \centering
\begin{tabular}{llll}
\toprule
{} & Mh Attn. Pool & Avg. Pool & Transformer \\
{} & No CLS & Int. Mh Attn. & Lr .0001 Drop. .1 \\
\midrule
Best Epoch          &                           21 &                  33 &                       50 \\
Eng.\&Img.r1  &                        9.55\% &               1.15\% &                    0.30\% \\
Eng.\&Img.r5  &                       27.85\% &               4.30\% &                    1.10\% \\
Eng.\&Img.r10 &                       38.50\% &               7.90\% &                    2.25\% \\
\midrule
Hin.\&Img.r1  &                        7.90\% &               1.10\% &                    0.25\% \\
Hin.\&Img.r5  &                       23.75\% &               3.90\% &                    0.90\% \\
Hin.\&Img.r10 &                       32.55\% &               6.50\% &                    2.15\% \\
\midrule
Jap.\&Img.r1  &                       16.55\% &               3.40\% &                    0.35\% \\
Jap.\&Img.r5  &                       43.15\% &              11.15\% &                    2.05\% \\
Jap.\&Img.r10 &                       57.50\% &              18.85\% &                    3.35\% \\
\bottomrule
\end{tabular}
\caption{Additional configurations for InfoNCE loss.}
\label{table:additional_obj_configs}
\end{table}

Results for image retrieval can be found in Table \ref{table:additional_obj_configs}, we omit the cross-lingual results for space concerns and redundat findings.

The recal scores indicate that removing the prepended token has a marginally negative impact on performance.
Somewhat surprisingly, though, the internal MHSA configuration and the Transformer block proved to subtantially harm retrieval.
This may be due to a lack of robust tuning, but it should be noted that average pooling has not hyperparameters itself and does not require tuning.




% \subsection{Triplet}
% \subsection{InfoNCE}
% \subsection{Masked Margin}
% \subsection{Scheduled Masked Margin}
% \section{Results}


%
\chapter{Loss Complexity}
\index{Loss Complexity}

Previous work in this area has noted that fully optimizing retrieval performance requires the tuning of each direction of each pair of views being optimized \citeme[Ask Dr. Harwath].
Since the number of pairings grows quadratically with each additional view, this can result in an untennable number of parameters to tune.
Furthermore, work by Chen et al. \cite{chen2020simple} and He et al. \cite{he2020momentum} have noted that state-of-the-art contrastive loss objectives are sensitive to the number of negative examples.
Since negative samples are commonly taken from within each mini-batch \cite{oord2018representation, ilharco2019large}, this means the best performing models need to be trained either using large batch sizes or a memory-bank mechanism with stale representations \cite{he2020momentum} \dftwrds[check that this is the correct citation].
It is as yet unclear how the increase in number of views/modalities effects this need for negative examples.
We hypothesize that as the number of additional views are added, all of which contain some aspect of the underlying latent information, the number of necessary negative examples might decrease.
These ideas motivate this chapter's experiments in which we explore strategies reduce the growth to a linear increase and assess the impact on performance.

We start with our experimental design in Section \ref{section:loss_comp_exp_design}, then we discuss our results in Section \ref{section:loss_comp_results}.

\section{Experimental Design}
\label{section:loss_comp_exp_design}
Most aspects of our experimental setup are identical to Section \ref{section:loss_comp_exp_design}, including the dataset and encoder architectures.
Again each view has a dedicated encoder and final representations are obtained by average pooling.
We use same Adam optimizer with a learning rate of .001 and InfoNCE loss.

Our main experimental variables involve the multi-view training framework outlined in Section \ref{section:backg_contrast_framework}.
Instead of using all view-pairs (i.e. the \textit{full-graph}), we try three variations of the \textit{anchor} framework.

In the first we simply use the image view as the anchor.
That is, we only compute the InfoNCE loss from view pairs that include the image view.
As the only view in the visual modality, this seems a natural choice.

In the second variation, we use an `average view' as the anchor, which is an average representation of all views.
Specifically, for each datapoint we average the final representations of each viewpoint (recall our definition of datapoint and viewpoint from Section \ref{section:task_defs}).
This effectively creates a centroid for each datapoint. % positively associated viewpoint.
Within the InfoNCE loss, our intuition is that positively associated viewpoints will be drawn closer to their own centroid, while being pushed away from centroids of other datapoints.

In the third variation, we deviate slightly from the multiview contrastive framework of \citeme and use an adapting the average view as our anchor.
This adapting average view changes according to which view is being contrasted.
To be concrete, we iterate over each view in a batch and for each view we use the average of all other views as the contrasting view for the InfoNCE loss.
This differs from our second variation in that instead of a fixed full average, which will contain in it the information from any view contrasted with it, we instead remove the information from the view being contrasted.

For each of the three variations we train a new set of encoders from scratch all under the same set of conditions other than the three variations described above.

\section{Resuts}
\label{section:loss_comp_results}
The recall at 1, 5, and 10 results for image retrieval for these three variation can be seen in Table \ref{table:complexity_opts_img_ret}.
\begin{table}
    \centering
\begin{tabular}{lllll}
\toprule
{Image Ret.} & Full-Graph & Image Anc. & Avg. Anc. & Cont. Others \\
\midrule
Best Epoch  &                      26 &                            23 &                   37 &                           16 \\
\midrule
E\&I.avgR1   &                  18.05\% &                       19.55\% &                0.10\% &                        13.05\% \\
H\&I.avgR1   &                  14.70\% &                       16.55\% &                0.00\% &                        12.30\% \\
J\&I.avgR1   &                  28.00\% &                       26.30\% &                0.05\% &                        25.20\% \\
\midrule
E\&I.avgR5   &                  41.55\% &                       44.00\% &                0.45\% &                        31.80\% \\
H\&I.avgR5   &                  36.10\% &                       37.85\% &                0.40\% &                        25.90\% \\
J\&I.avgR5   &                  59.20\% &                       58.90\% &                0.35\% &                        55.55\% \\
\midrule
E\&I.avgR10  &                  54.55\% &                       57.65\% &                0.80\% &                        44.55\% \\
H\&I.avgR10  &                  45.85\% &                       47.10\% &                0.60\% &                        34.60\% \\
J\&I.avgR10  &                  72.35\% &                       72.70\% &                0.75\% &                        67.95\% \\
\bottomrule

\end{tabular}
\caption{Loss complexity reduction strategies. Image retrieval results.}
\label{table:complexity_opts_img_ret}
\end{table}

Somewhat surprisingly, image retieval performance increased when removing the cross-lingual objective from the loss.
In one sense, it would seem natural that solely focusing on the image retrieval task in the loss function should produce beter image retrieval scores.
However, when viewed from the lens of shared mutual information, one might expect that, at an abstract level at least, removing one language encoder's incentive to learn associations with other languages would harm the model's ability to properly encode speech into a space that is shared by all languages and images.

This indicates that there might be some language or cultually specific tendencies present in each language view that co-occur consistently with certain aspects of the image set that have little to no co-occurence with the descriptions of other languages.
Considering the results from Harwath et al. \cite{} and Ohishi et al. \cite{ohishi2020trilingual}, it seems likely that there is a benifit from a shared multi-lingual


\begin{table}
    \centering
\begin{tabular}{llll}
\toprule
{} & Cont. Others & Avg. Anc. & Image Anc. \\
\midrule
Best Epoch          &                            23 &                   37 &                           16 \\
\midrule
hin\&eng\_dot\_avg\_r1  &                         8.45\% &                0.05\% &                        0.95\% \\
hin\&eng\_dot\_avg\_r5  &                        21.70\% &                0.35\% &                        4.40\% \\
hin\&eng\_dot\_avg\_r10 &                        30.75\% &                0.70\% &                        6.85\% \\
\midrule
jap\&eng\_dot\_avg\_r1  &                         8.65\% &                0.15\% &                        1.40\% \\
jap\&eng\_dot\_avg\_r5  &                        25.15\% &                0.60\% &                        4.55\% \\
jap\&eng\_dot\_avg\_r10 &                        35.35\% &                1.80\% &                        6.70\% \\
\midrule
jap\&hin\_dot\_avg\_r1  &                         7.35\% &                0.20\% &                        1.15\% \\
jap\&hin\_dot\_avg\_r5  &                        19.25\% &                0.45\% &                        4.05\% \\
jap\&hin\_dot\_avg\_r10 &                        27.80\% &                1.00\% &                        6.25\% \\
\bottomrule
\end{tabular}
\caption{Loss complexity reduction strategies. Cross-lingual retrieval results.}
\label{table:complexity_opts_cling_ret}
\end{table}



%

\chapter{Parameter Efficiency}
\index{Parameter Efficiency}

\section{Results}
\begin{table}
    \centering
    \begin{tabular}{llll}
    \toprule
    {} & Basic & Chan. Dim. Tok. & Seq Dim. Tok. \\
    \midrule
    Best Epoch          &                     19 &                        11 &                            17 \\
    eng\&img\_r1  &                 11.35\% &                     2.30\% &                         7.35\% \\
    eng\&img\_r5  &                 31.05\% &                     8.40\% &                        22.50\% \\
    eng\&img\_r10 &                 42.70\% &                    12.80\% &                        32.10\% \\
    hin\&img\_r1  &                  7.40\% &                     2.20\% &                         4.40\% \\
    hin\&img\_r5  &                 21.20\% &                     8.65\% &                        14.10\% \\
    hin\&img\_r10 &                 29.70\% &                    13.45\% &                        20.20\% \\
    jap\&img\_r1  &                  8.60\% &                     2.25\% &                         4.60\% \\
    jap\&img\_r5  &                 25.85\% &                     7.60\% &                        16.45\% \\
    jap\&img\_r10 &                 35.60\% &                    12.25\% &                        23.20\% \\
    \bottomrule
    \end{tabular}
\caption{Parameter efficiency strategies. One encoder for all languages. Image retrieval resutls.}
\label{table:param_eff_img_ret}
\end{table}

\begin{table}
    \centering
    \begin{tabular}{llll}
    \toprule
    {} & Basic & Chan. Dim. Tok. & Seq Dim. Tok. \\
    \midrule
    Best Epoch          &                     19 &                        11 &                            17 \\
    hin\&eng\_dot\_avg\_r1  &                  2.25\% &                     0.20\% &                         1.55\% \\
    hin\&eng\_dot\_avg\_r5  &                  8.35\% &                     1.55\% &                         6.85\% \\
    hin\&eng\_dot\_avg\_r10 &                 14.10\% &                     3.10\% &                        11.00\% \\
    jap\&eng\_dot\_avg\_r1  &                  2.80\% &                     0.30\% &                         1.75\% \\
    jap\&eng\_dot\_avg\_r5  &                 10.05\% &                     1.50\% &                         6.60\% \\
    jap\&eng\_dot\_avg\_r10 &                 15.45\% &                     2.95\% &                        10.00\% \\
    jap\&hin\_dot\_avg\_r1  &                  1.35\% &                     0.15\% &                         1.50\% \\
    jap\&hin\_dot\_avg\_r5  &                  5.80\% &                     1.20\% &                         5.45\% \\
    jap\&hin\_dot\_avg\_r10 &                 10.40\% &                     1.65\% &                         8.45\% \\
    \bottomrule
    \end{tabular}
\caption{Parameter efficiency strategies. One encoder for all languages. Cross-lingual retrieval resutls.}
\label{table:param_eff_cling_ret}
\end{table}



%
\include{contents/chapter-conclusion}


%%%%%%%%%%%%%%%%%%%%%%%%%%%%%%%%%%%%%%%%%%%%%%%%%%%%%%%%%%%%%%%%%%%%%%
% Appendix/Appendices                                                %
%%%%%%%%%%%%%%%%%%%%%%%%%%%%%%%%%%%%%%%%%%%%%%%%%%%%%%%%%%%%%%%%%%%%%%
%
% If you have only one appendix, use the command \appendix instead
% of \appendices.
%
\appendices
\index{Appendices@\emph{Appendices}}%

\include{orig_template_contents/chapter-appendix1}

% \include{orig_template_contents/chapter-appendix2}
%
% \include{orig_template_contents/chapter-appendix3}


%%%%%%%%%%%%%%%%%%%%%%%%%%%%%%%%%%%%%%%%%%%%%%%%%%%%%%%%%%%%%%%%%%%%%%
% Generate the bibliography.					     %
%%%%%%%%%%%%%%%%%%%%%%%%%%%%%%%%%%%%%%%%%%%%%%%%%%%%%%%%%%%%%%%%%%%%%%
%								     %
% NOTE: For master's theses and reports, NOTHING is permitted to     %
%	come between the bibliography and the vita. The command      %
%	to generate the index (if used) MUST be moved to before      %
%	this section.						     %
%								     %
\nocite{*}      % This command causes all items in the 		     %
                % bibliographic database to be added to 	     %
                % the bibliography, even if they are not 	     %
                % explicitly cited in the text. 		     %
		%						     %
\bibliographystyle{plain}  % Here the bibliography 		     %
\bibliography{diss}        % is inserted.			     %
\index{Bibliography@\emph{Bibliography}}%			     %
%%%%%%%%%%%%%%%%%%%%%%%%%%%%%%%%%%%%%%%%%%%%%%%%%%%%%%%%%%%%%%%%%%%%%%


%%%%%%%%%%%%%%%%%%%%%%%%%%%%%%%%%%%%%%%%%%%%%%%%%%%%%%%%%%%%%%%%%%%%%%
% Generate the index.						     %
%%%%%%%%%%%%%%%%%%%%%%%%%%%%%%%%%%%%%%%%%%%%%%%%%%%%%%%%%%%%%%%%%%%%%%
%								     %
% NOTE: For master's theses and reports, NOTHING is permitted to     %
%	come between the bibliography and the vita. This section     %
%	to generate the index (if used) MUST be moved to before      %
%	the bibliography section.				     %
%								     %
\printindex%    % Include the index here. Comment out this line      %
%		% with a percent sign if you do not want an index.   %
%%%%%%%%%%%%%%%%%%%%%%%%%%%%%%%%%%%%%%%%%%%%%%%%%%%%%%%%%%%%%%%%%%%%%%


%%%%%%%%%%%%%%%%%%%%%%%%%%%%%%%%%%%%%%%%%%%%%%%%%%%%%%%%%%%%%%%%%%%%%%
% Vita page.							     %
%%%%%%%%%%%%%%%%%%%%%%%%%%%%%%%%%%%%%%%%%%%%%%%%%%%%%%%%%%%%%%%%%%%%%%

\begin{vita}
Craig William McCluskey
was born in Minneapolis, Minnesota on 20 May 1950, the son of
Dr. William R. McCluskey and Lucilla W. McCluskey.  He received the Bachelor
of Science degree in Engineering from the California Institute of Technology
and was commissioned an Officer in the United States Air Force in 1971.
He entered active duty in October, 1971, and was stationed in Denver, Colorado,
Colorado Springs, Colorado, Panama City, Florida, and Sacramento, California.
He separated from the USAF in 1975 and worked as an engineer for several small
electronics companies in California before moving to Colorado Springs, Colorado
to work for Hewlett-Packard in 1979. He left Hewlett-Packard in 1989 and joined
a small company based in Herndon, Virginia, working out of his house as a
``remote'' engineer designing parts of the Alexis satellite for Los Alamos
National Laboratories. Laid off when his portion of the satellite was
completed, he applied to the University of Texas at Austin for enrollment
in their physics program. He was accepted and started graduate studies in
August, 1991.

\end{vita}

\end{document}
